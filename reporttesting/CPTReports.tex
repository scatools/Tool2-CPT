\documentclass[]{article}
\usepackage{lmodern}
\usepackage{amssymb,amsmath}
\usepackage{ifxetex,ifluatex}
\usepackage{fixltx2e} % provides \textsubscript
\ifnum 0\ifxetex 1\fi\ifluatex 1\fi=0 % if pdftex
  \usepackage[T1]{fontenc}
  \usepackage[utf8]{inputenc}
\else % if luatex or xelatex
  \ifxetex
    \usepackage{mathspec}
  \else
    \usepackage{fontspec}
  \fi
  \defaultfontfeatures{Ligatures=TeX,Scale=MatchLowercase}
\fi
% use upquote if available, for straight quotes in verbatim environments
\IfFileExists{upquote.sty}{\usepackage{upquote}}{}
% use microtype if available
\IfFileExists{microtype.sty}{%
\usepackage{microtype}
\UseMicrotypeSet[protrusion]{basicmath} % disable protrusion for tt fonts
}{}
\usepackage[margin=1in]{geometry}
\usepackage{hyperref}
\hypersetup{unicode=true,
            pdftitle={Grand Bay   Conservation Prioritization Tool Report},
            pdfauthor={Strategic Conservation Assessment Project},
            pdfborder={0 0 0},
            breaklinks=true}
\urlstyle{same}  % don't use monospace font for urls
\usepackage{graphicx,grffile}
\makeatletter
\def\maxwidth{\ifdim\Gin@nat@width>\linewidth\linewidth\else\Gin@nat@width\fi}
\def\maxheight{\ifdim\Gin@nat@height>\textheight\textheight\else\Gin@nat@height\fi}
\makeatother
% Scale images if necessary, so that they will not overflow the page
% margins by default, and it is still possible to overwrite the defaults
% using explicit options in \includegraphics[width, height, ...]{}
\setkeys{Gin}{width=\maxwidth,height=\maxheight,keepaspectratio}
\IfFileExists{parskip.sty}{%
\usepackage{parskip}
}{% else
\setlength{\parindent}{0pt}
\setlength{\parskip}{6pt plus 2pt minus 1pt}
}
\setlength{\emergencystretch}{3em}  % prevent overfull lines
\providecommand{\tightlist}{%
  \setlength{\itemsep}{0pt}\setlength{\parskip}{0pt}}
\setcounter{secnumdepth}{0}
% Redefines (sub)paragraphs to behave more like sections
\ifx\paragraph\undefined\else
\let\oldparagraph\paragraph
\renewcommand{\paragraph}[1]{\oldparagraph{#1}\mbox{}}
\fi
\ifx\subparagraph\undefined\else
\let\oldsubparagraph\subparagraph
\renewcommand{\subparagraph}[1]{\oldsubparagraph{#1}\mbox{}}
\fi

%%% Use protect on footnotes to avoid problems with footnotes in titles
\let\rmarkdownfootnote\footnote%
\def\footnote{\protect\rmarkdownfootnote}

%%% Change title format to be more compact
\usepackage{titling}

% Create subtitle command for use in maketitle
\newcommand{\subtitle}[1]{
  \posttitle{
    \begin{center}\large#1\end{center}
    }
}

\setlength{\droptitle}{-2em}

  \title{\textbf{Grand Bay} Conservation Prioritization Tool Report}
    \pretitle{\vspace{\droptitle}\centering\huge}
  \posttitle{\par}
    \author{\emph{Strategic Conservation Assessment Project}}
    \preauthor{\centering\large\emph}
  \postauthor{\par}
      \predate{\centering\large\emph}
  \postdate{\par}
    \date{April 16, 2019}


\begin{document}
\maketitle

\subsection{Spatial footprint of conservation area
assessed}\label{spatial-footprint-of-conservation-area-assessed}

\subsubsection{\texorpdfstring{\textbf{MAP GOES
HERE}}{MAP GOES HERE}}\label{map-goes-here}

\begin{itemize}
\tightlist
\item
  {[} {]} Mercury
\item
  {[}x{]} Venus
\item
  {[}x{]} Earth (Orbit/Moon)
\item
  {[}x{]} Mars
\item
  {[} {]} Jupiter
\item
  {[} {]} Saturn
\item
  {[} {]} Uranus
\item
  {[} {]} Neptune
\item
  {[} {]} Comet Haley
\end{itemize}

\subsection{\texorpdfstring{\textbf{Summary}}{Summary}}\label{summary}

This report evaluates the \textbf{Grand Bay} area of interest,
approximately \textbf{32 acres} of land adjacent to the 2000, and
houses\\
1, NA, and NA (5 percent of the project area). Lands within Grand Bay
support a diversity of fish and wildlife, including the federally listed
1 and NA, and contains roughly 29 percent of the NA's critical habitat.
Also, protection of Grand Bay would preserve a considerable amount of
structural connectivity to surrounding lands, as 1 percent of the
project area is classified as a hub or corridor by the EPA National
Ecological Framework. \textbf{Conserving this area of interest would
protect longleaf pine and rangeland working lands (about 31 percent of
the project area).} This area of interest also buffers water flowing
into waterbodies with known impairments (McCrae Dead River and
Tchoutacabouffa River) and preservation would allow this landscape to
continue to provide such water quality protections. In the future, the
Grand Bay area may be vulnerable to inundation due to sea level rise,
and is expected to have a high threat of development by the year 2060
according to the SLEUTH urbanization model.

\subsubsection{\texorpdfstring{\textbf{Data Table Goes
Here}}{Data Table Goes Here}}\label{data-table-goes-here}

\subsubsection{Any Additional Figures ???}\label{any-additional-figures}

\subsection{\texorpdfstring{\textbf{Supporting
Information}}{Supporting Information}}\label{supporting-information}

\paragraph{Definitions of Raw Data
Measures}\label{definitions-of-raw-data-measures}

\begin{enumerate}
\def\labelenumi{\arabic{enumi}.}
\tightlist
\item
  \textbf{Threat of Urbanization -} Threat of Conversion indicates the
  likelihood of the proposed conservation area to be urbanized by the
  year 2060. A score of zero indicates the hexagon is already urban and
  score of 0+ to one indicates the predicted likelihood of threat in
  decreasing order. A score of one indicates absolutely no threat of
  conversion based on SLEUTH 2060 urbanization model.
\item
  \textbf{Connectivity with PAD-US -} Connectivity to PAD-US indicates
  of the proposed conservation area is close to an area classified as
  protected by PAD-US data. A binary attribute which represents the
  spatial relationship between Hexagon and PAD-US. Any Hexagon directly
  intersects or within 1 Hex (1 km2) distance would be count as 1,
  otherwise, 0.
\item
  \textbf{Structural Connectivity -} Hub and Corridors A percent
  attribute which stands for the proportion of area, been classified as
  Hub or Corridor by the raw data source, within each Hexagon. Since the
  Hexagon unit area is 1 Km2, it also stands for the actual area of Hub
  within each Hexagon.
\item
  \textbf{Proposed Area of Conservation -} The area of the proposed area
  of interest, in acres.
\item
  \textbf{303(d) - Impaired Watershed Area -} A percent attribute which
  stands for the proportion of impaired watershed within each Hexagon.
  The watershed data is analyzed based on HUC12 level. Any HUC12
  watershed contains 303(d) impaired streams would be considered as
  impaired.
\item
  \textbf{Vulnerable Areas of Terrestrial Endemic Species -} A zero score 
  indicates the lowest biodiversity and score of 0+ to 10 indicates 
  biodiversity in increasing order. A score of 10 indicates highest 
  biodiversity within Gulf of Mexico region. Vulnerable Areas of Terrestrial 
  Endemic Species were classified into 10 groups based on the same method 
  proposed in Jenkins (2015)\footnote{Jenkins,
    CN, KS Van Houtan, SL Pimm, JO Sexton (2015) US protected lands
    mismatch biodiversity priorities. PNAS 112(16), pp.5081-5086.}.
\item
  \textbf{T\&E Species Area -} The attribute is based on the U.S. Fish
  \& Wildlife Service designated T\&E critical habitat. The value in
  each hexagon is the cumulative \% area of critical habitats for all
  T\&E species.
\item
  \textbf{T\&E Species Counts -} A numeric attribute which represents
  the number of T\&E Species within each Hexagon. The attribute is based
  on the U.S. Fish \& Wildlife Service designated T\&E critical habitat.
\item
  \textbf{Light Pollution Index -} A score of zero indicates the sky
  above the hex is already polluted/bright and score of 0+ to one
  indicates light pollution (LP) in decreasing order.
\item
  \textbf{National Register of Historic Places -} A numeric attribute
  which represents the counts of historical Places within each Hexagon.
  The data is based on U.S. NPS National Register of Historic Places.
\item
  \textbf{National Heritage Area -} A \% attribute which stands for the
  proportion of Heritage area within each Hexagon. The Heritage data is
  based on the NPS National Heritage Area layer.
\item
  \textbf{Land Cover - High Priority -} The total \% area of identified
  top priority land cover (Tier 1) classes within a hexagon created from
  NCLD, CCAP, and GAP land cover classification maps.
\item
  \textbf{Working Lands -} High Priority The \% area of Pine, Cropland
  and Pasture/Hay classes from NLCD classification map excluding the
  areas that are already protected (PAD-US).
\item
  \textbf{Commercial Fisheries Reliance -} Engagement Commercial fishing
  engagement measures the presence of commercial fishing through fishing
  activity as shown through permits and vessel landings. A high rank
  indicates more engagement.
\item
  \textbf{Recreational Fisheries Engagement -} Engagement Recreational
  fishing engagement measures the presence of recreational fishing
  through fishing activity estimates. A high rank indicates more
  engagement.
\end{enumerate}


\end{document}
